% Chapter 5
\chapter{Summary and Future Works}
\label{chap:conclusion}

\section{Summary}

The purpose of this research topic is to implement a system evaluating a star tracking algorithm proposed by Mr. Minh Duc Pham\cite{MDP,edselc.2-52.0-8487677997120120101,edseee.655799920130101}, my senior and an ex post-graduated NTU student. In my thesis, I have summarized three traditional methods and one state-of-the-art star tracking method proposed in the literature review section. Then in the Hardware and Software Co-processing – Implementation of the Algorithm section, I introduced the hardware system, deeply analyze the algorithm. I also proposed an adaptive algorithm based on the proposed algorithm that is suitable for the software-hardware co-processing implementation in the system. In the final section Experimental Results, I suggested my hardware architecture design for this project, carry out some experiments to evaluate the implementation and their results. \\

\noindent All of my work includes the simulation image generator, image datasets, hardware IP core design, system architecture, software algorithms and source code, results of the experiments, quick documentations are dedicated in this repository: https://github.com/dangkhoadl/Master-Thesis-Star-Tracking-System

\section{Future Works}

Currently, the system can not handle a high throughput image data streaming. In my experiments, I have to handle the images one-by-one to feed into the system. Hence, I suggest implementing an input-processing module with can manage a large stream amount of image data(around 9.8 GBs) autonomously. \\

\noindent Moreover, the first dataset is generated by sky simulation with slightly normal distribution noise added, the second dataset is captured by a star tracker from the ground. In space, there are more noises affect the quality of the images from the electromagnetic field, Solar noise, Cosmic noise, etc. A noise cancellation module implemented in the PL would be needed.
