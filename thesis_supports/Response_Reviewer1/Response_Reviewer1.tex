\documentclass[dvips,a4paper,12pt]{report}
\usepackage{xcolor}
\usepackage[
    top=0.6in,
    bottom=0.6in,
    right=0.6in,
    left=0.6in]{geometry}

\begin{document}
\pagenumbering{arabic}


% ----------------------------------------- Begin content --------------------------------------------- %
%%% letter
\noindent Respected Examiner,

\noindent Thank you for reviewing my thesis.

\begin{center}
    \textbf{Thesis Title: Implementation of an Autonomous Star Recognition Algorithm using Hardware-Software Co-processing Approach} 
\end{center}

\noindent Firstly, I would like to thank you for your insightful comments. These excellent comments have helped significantly improve my thesis. This letter addresses each of the raised points and notes precisely how I have responded and included the comments into my revised thesis. \\

\vspace{0.5in}

\noindent Sincerely yours,

\noindent Dang Le Dang Khoa

%%% Q&A
\newpage

\color{blue}
\noindent \large{\textbf{Examiner: 1}}

\begin{enumerate}
    \color{blue}
    \item There are some obvious typos and format problems in the thesis. For example, in Page 7, the caption ‘table 2.1’ is written, but the whole table is given in Page 8. Please check and correct them.

    \color{black}
    Answer

    \color{blue}
    \item In this thesis, the author shows some figures such as Fig. 2.2, 2.3 without any explanation. Please add some explanation to these figures in the thesis content.

    \color{black}
    Answer

    \color{blue}
    \item In Chapter 3.3.1, the author applied thresholding technique in order to eliminate the noise and separate star clusters from the background. But the author did not explain how to choose the threshold.

    \color{black}
    Answer

    \color{blue}
    \item The author claimed that the proposed prebuilt tree structured star pattern database (SPD) was able to reduce space complexity while maintaining high accuracy and robustness, so this optimized algorithm could be used for nanosatellites which have limited memory. However, there is no comparison between the SPD algorithm and traditional algorithm on accuracy, storage consumption, and robustness in this thesis.

    \color{black}
    Answer

    \color{blue}
    \item In Chapter 4.2, the author compares runtime spent by traditional methods and the proposed method in this thesis. He lists specific software runtime and hardware – software co-processing runtime when the same image dataset is used. Details of images used in experiments are also provided. However, the author only lists tables and figures to show image specifications and runtime. He does not give even one word to explain or conclude his experiment results.

    \color{black}
    Answer

    \color{blue}
    \item In Chapter 4.1, the author gives experiment results showing hardware consumption and power consumption when the proposed method is applied. But he does not give comparison on consumption between traditional methods and his own method, therefore, readers do not know how to evaluate his experiment results. If he could add experiment results about power and hardware consumption when traditional methods are used (under the same experimental conditions), and then give a comparison between traditional methods and his method, the advantages of his proposed method would be clearer.

    \color{black}
    Answer

\end{enumerate}

% ----------------------------------------- End content --------------------------------------------- %
\end{document}
